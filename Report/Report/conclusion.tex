% !TEX root = ../main.tex
% !TEX spellcheck = en_GB

\chapter{Conclusion}
In summary the implementation of Little helper on the black fin wasn't as for filling as the goal of the project.
The project algorithm had to be changed when the limitations of the black fin was found with regards to the ability to change the fs.
Based on this experience, it is necessary to have a good understanding of the processor which one is working on. 
So the developer have the knowledge of the limitations as well the strength of the processor. 
With this information the developer can easily see or predict a future hurdle with the algorithm.

The synergy between Matlab and the implementation on the black fin, is absolutely unimaginable brilliant.
Both in regards to the development of the algorithm, as well as the debugging and implementation on the processor. 
A troublesome thing, with writing the algorithm in Matlab is, that without the sufficient knowledge of processor, the developer might make an algorithm which cannot or is badly optimized on the processor.
If that is the case the developer has to go back to drawing board, and make a new iteration.

The final algorithm which uses the Hilbert transform is very impressive, with the high frequency resolution, and the surprisingly fast computation of a signal block.
Of course it has it's constraints with regards to the problem when more sins are mixed together as a tone from a piano or a guitar is.
The next part of the system, the sine generator, wasn't fully implemented so the effectiveness or constraints to this part is a bit unknown, but it works in Matlab.
The issues there might be with speed, memory space and quantization isn't present in Matlab, but are highly important on the processor.
So it would have been really interesting to implement the next parts of the system to discover if there are fatal issues or if it is a flawless system.

To conclude the project has been formed over frequent iterations and many ideas, with many hurdles and the outcome of a partly working system on the black fin and a full functional implementation in Matlab.
  
\FloatBarrier