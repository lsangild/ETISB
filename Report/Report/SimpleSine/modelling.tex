% !TEX root = ../../main.tex
% !TEX spellcheck = en_GB

\section{Modelling}
\label{sec:PureSineModelling}
\subsection{Preprocessing}
The preprocessing in MATLAB takes the FFT of the incoming signal, finds the largest peak in the spectrum and determines the ratio between the input and the goal frequency.
The ratio is used to determine how to change the frequency, as shown in \cref{eq:1ratio}.

\begin{equation}
	f_{s,new} = f_{s,in}\cdot\frac{f_{new}}{f_{in}}
	\label{eq:1ratio}
\end{equation}

\subsection{Signal processing}
If the new sampling frequency, determined in preprocessing, is lower than the old, samples are cut from the end of the signal. \fxnote{figurer? MATLAB kode?}
If the new sampling frequency is higher than the old the phase at which the input signal ends is determined.

\subsection{Post processing}
If the input is lower than the output sampling frequency, the found phase is used to add a tail to the signal with the phase offset, to create an output signal of equal time as the input signal.

\FloatBarrier