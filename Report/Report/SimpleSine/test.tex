% !TEX root = ../../main.tex
% !TEX spellcheck = en_GB

\section{Test}
\label{sec:test}
In this section an short introduction to how this project has used test from beginning to the end will be described.
\subsection{Frequency determination resolution}
\begin{table}
	\fxnote{better description or table view, i dont get it atm.}
	\centering
	\begin{tabular}{c | c c c c c | c}
		\toprule
		Test freq (Hz) & \multicolumn{4}{l}{Matlab freq (Hz)} & & CC freq (Hz) \\
		\midrule
		\num{245} & \num{245.13} & \num{244.93} & \num{244.85} & \num{245.03} && \num{237}\\
		\num{350} & \num{349.90} & \num{349.94} & \num{349.88} & \num{349.96} && \num{349}\\
		\num{470} & \num{470.00} & \num{470.00} & \num{470.00} & \num{470.00} && \num{470} \\
		\num{521} & \num{521.22} & \num{521.18} & \num{521.10} & \num{521.04} && \num{525} \\
		\bottomrule
	\end{tabular}
	\caption{Tested frequencies and the found results using blocks of length \num{512}. Matlab results are the average of 4 blocks of \num{512} on a signal of length \num{2048}, each block windowed with a hamming window.}
	\label{tab:test}
\end{table}

\subsection{Clock cycle usage}
A question to be asked is, "Can the processing of the signal occur before the next set of 512 samples?".
To find the answer for this, the clock cycles used of the FFT an the hilbert transform has been measured, and the amount of time at disposal has been calculated.

Time between sample blocks:
$td = \frac{samples}{fs} = \frac{512}{\SI{48}{\kilo\hertz}} = \SI{10.7}{\milli\second}$

Measurement of the clock cycles usage:
$\frac{1898733 cycles}{512 samples} = 3708 \frac{cyles}{sample}$.
Which is less then specified in the requirements.

The Blackfin has a maximum speed at \SI{600}{\mega\hertz}.
Which means it will take $\frac{\num{1898733}}{\SI{600}{\mega\hertz}} = {\SI{3.28}{\milli\second}}$ so the Hilbert transformation should not take more time than it has at its disposal.\fxnote{Better explanation needed}

To generate 183 samples, with the sine generator it takes: $\frac{26225 cycles}{183 samples} = 143.3\frac{cycles}{sample}$. 

\subsubsection{Optimized clock cycle usage}
In the cross core environment it is possible to optimize the code with regards to size or speed.
This was done with three different settings for the optimization. 
Firstly where the size was the main focus of the optimization. 
Secondly the size and speed had same prioritization in the optimization settings.
And lastly where the speed was the focus. The result can be seen on
\begin{table}
	\centering
	\caption{The cycle usage with different settings for the optimization function in the cross core environment}
	\label{OptimizedCycleUsage}
	\begin{tabular}{l|l|l|l}
		\hline
		{\textbf{Cycle usage pr.sample for:}} & \textbf{Size} & \textbf{50\% Size 50\% Speed} & \textbf{Speed} \\ \hline
		\textbf{Hilbert transform}                &  $\frac{1833887 cycles}{512 samples} = 3582 \frac{cycles}{sample} $           &    $\frac{1831492 cycles}{512 samples} = 3577 \frac{cycles}{sample}$                            &   $\frac{181819687 cycles}{512 samples} = 3554 \frac{cycles}{sample}$             \\ \hline
		\textbf{Sine generator}                   &       $\frac{21085 cycles}{183 samples} = 115 \frac{cycles}{sample}$        &    $\frac{21084 cycles}{183 samples} = 115 \frac{cycles}{sample}$                           &    $\frac{21082 cycles}{183 samples} = 115 \frac{cycles}{sample}$ 	\\ \hline 
		\textbf{Total}    & 	$3697 \frac{cycles}{sample}$	 & 	$3692 \frac{cycles}{sample}$	&	 $3669 \frac{cycles}{sample}$	\\ \hline 
	\end{tabular}
\end{table}

So if the code is optimized with regards to speed, the amount of cycles used becomes less, but a neglect able amount.  

\FloatBarrier