% !TEX root = ../../main.tex
% !TEX spellcheck = en_GB

\section{Implementation}
\label{sec:impl}
The implementation has been done in the framework made by our supervisor Kim Bjerge. The complete idea of the system has not been implemented on the blackfin. What has been implemented on the blackfin is the Hilbert transformation, which determines the main frequency in the signal and a version of the sine generator for test purpose. 
To the framework has been added a function called hilbert() and can be seen below and a function called sineGenerator(). Before the hilbert() is called the FFT of the signal will be made using the function 
\begin{minted}{python}
void rfft_fr16(const fract16          input[],
complex_fract16        output[],
const complex_fract16  twiddle_table[],
int                    twiddle_stride,
int                    fft_size,
int                   *block_exponent,
int                    scale_method);
\end{minted}
After the FFT, the hilbert() will use the ftt signal. In hilbert() the fft signal will first be rearranged into the array $zHilbert[N\_FFT]$. The first value and the $N\_FFT/2$ value will be waited less and therefore are multiplied with $0.5$ using the optimized function $cmlt\_fr16(a,b)$. 

From $N\_FFT/2$ and beyond everything is set to zero. 
After rearranging the inverse FFT of zHilbert, is made and placed in $m\_ifft\_output$. The inverse FFT is also made using the build in functions.
\begin{minted}[linenos, breaklines, bgcolor=lightgray]{c}
void DynamicFilter::hilbert(){
	// parameters
	int block_exponent;
	fract16 instafreq[N_FFT];
	fract16 meanFreq;
	fract32 temp = 0;
	uint16_t dif;
	complex_fract16 scale;
	scale.re=0.5; scale.im=0.5;
	fract32 temp32[512];
			
	// Hilbert transform
	zHilbert[0] = cmlt_fr16(m_fft_output[0], scale);
	
	for (int i = 1; i<N_FFT/2; i++){
		zHilbert[i] = m_fft_output[i];
	}
		
	zHilbert[N_FFT/2] = cmlt_fr16(m_fft_output[N_FFT/2], scale);
	
	// ifft of the hilbert signal
	ifft_fr16(zHilbert, m_ifft_output, m_twiddle_table, 1, N_FFT, &block_exponent, 1);
		
	////// instant frequency finding //////
	//first finding the phase
	for (int i = 0; i<N_FFT; i++){
		instafreq[i] = atan2_fr16(m_ifft_output[i].im, m_ifft_output[i].re);
	}
	
	// unwrapping with threshold 2PI
	temp32[0] = instafreq[0];
	for(int i = 1; i<N_FFT;i++){
		dif = (((int16_t)instafreq[i - 1] - (int16_t)instafreq[i]) + (1 << 15)) % (2 * (1 << 15)); //1<<15 = pi
		
		temp32[i] = ((int32_t)(dif - (1 << 15)));
			
		temp32[i] = sub_fr1x32(temp32[i-1],	temp32[i]); // optimering
	}
		
	// find the diff
	for(int i = 1; i<N_FFT;i++){
		temp32[i-1] = (sub_fr1x32(temp32[i],temp32[i-1])) * fs/(2*PI_FLOAT); //optimering
	}
		
	//finding the mean and therefore the value
	for(int i = 100; i<N_FFT-99;i++){
		temp += temp32[i];
	}
		
	meanFreq = ((temp/(N_FFT-200)) * PI_FLOAT) / (1<<15);
		
}
\end{minted}
Now the instant frequency finding starts. First the argument of every value of the IFFT signal is found using $atan2\_fr16$. Afterwards the remainder is found after the subtraction of two arguments. lastly the difference is found and then a mean of the middle values are used to find the meanFreq.

\subsection{Optimized functions}
In the implementation has some of the build in function been used, these functions are optimized and should use as few clock cycles as possible to execute their task. On \cref{tab:optimized_func}  some of the different build-in functions which are used can be seen.
\begin{table}
	\centering
	\begin{tabular}{l l}
		\toprule
		function name & what it does\\
		\midrule
		\begin{tabular}[c]{@{}l@{}}$rfft\_fr16(input, output, twiddle\_table, twiddle\_stride,$\\ $fft\_size, *block\_exponent, scale\_method)$\end{tabular}& makes the FFT of an input signal\\
		\begin{tabular}[c]{@{}l@{}}$ifft\_fr16(input, output, twiddle\_table, twiddle\_stride,$\\ $fft\_size, *block\_exponent, scale\_method)$\end{tabular}& makes the inverse FFT of an input signal\\
		$cmlt\_fr16(a,b)$& Multiply two complex fract16\\
		$sub\_fr1x32(a,b)$& subtract two complex fract32\\
		$atan2\_fr16(imaginary\_part,real\_part)$& findes the argument of a complex fract16 \\
		\bottomrule
	\end{tabular}
	\label{tab:optimized_func}
\end{table}

$twiddle\_table$ is a table made in the setup of the program. It is $W_n = cos(\frac{2\pi kn}{N})-jsin(\frac{2\pi kn}{N})$. It is made so that the FFT and IFFT are executed faster since these function don't have to calculate this part of the equation but can just take the value from the $twiddle\_table$. As long as the size of the FFT or IFFT doesn't change these values in the twiddle factor don't have to be calculated again.


\FloatBarrier