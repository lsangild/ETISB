% !TEX root = ../main.tex
% !TEX spellcheck = en_GB

\chapter{Introduction}
\label{sec:introduction}
We all know the feeling when your favorite song appears in the radio, and you can't stop yourself from singing along, even though you know, and everyone knows that it sounds like a cat fight.
\systemName is going to make even the most horrifying singer into a decent singer.
The goal of \systemName is to help the singer, so if the singer can't hit a tone, \systemName will help with this. \systemName will change the incoming sound and change it to the nearest tone, that way the user will never have a problem hitting an A' again.
But this is not all \systemName can accomplish. In today's modern Danish rap music, it is very popular to use auto tune in a way, so it is obvious.
Therefore \systemName will have the option to 'over' auto tune the incoming sound if the user wishes it.
This way even a person with no singing skills, can become a rapper, just with a little help from little helper.

\section{Project Description}
\label{sec:projectDescription}


\section{Project delimitations}
\label{sec:delimitations}
The \nameref{sec:projectDescription} is the groups vision of the ideal solution to the problem outlined in the \nameref{sec:introduction}.  
To give the best possible view of the groups capabilities in developing such a solution, the most core parts of the project will have the highest priority, hence some parts will be excluded from this version of the project. 

\section{Terminology}
\label{sec:terminology}
Below in \cref{tab:terminology} is shown a list of terms used throughout the report describing each name for clarification purposes.

\begin{table}[H]
	\centering
	\begin{tabularx}{0.8\textwidth}{l X}
		\toprule
		\textbf{Name} & \textbf{Description} \\
		\midrule
		Resolution & \\
		Scale & \\
		\systemName & The systems name. \\
		\bottomrule
	\end{tabularx}
	\caption{List of terminologies.}
	\label{tab:terminology}
\end{table}